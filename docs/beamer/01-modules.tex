% Options for packages loaded elsewhere
\PassOptionsToPackage{unicode}{hyperref}
\PassOptionsToPackage{hyphens}{url}
%
\documentclass[
  ignorenonframetext,
]{beamer}
\usepackage{pgfpages}
\setbeamertemplate{caption}[numbered]
\setbeamertemplate{caption label separator}{: }
\setbeamercolor{caption name}{fg=normal text.fg}
\beamertemplatenavigationsymbolsempty
% Prevent slide breaks in the middle of a paragraph
\widowpenalties 1 10000
\raggedbottom
\setbeamertemplate{part page}{
  \centering
  \begin{beamercolorbox}[sep=16pt,center]{part title}
    \usebeamerfont{part title}\insertpart\par
  \end{beamercolorbox}
}
\setbeamertemplate{section page}{
  \centering
  \begin{beamercolorbox}[sep=12pt,center]{part title}
    \usebeamerfont{section title}\insertsection\par
  \end{beamercolorbox}
}
\setbeamertemplate{subsection page}{
  \centering
  \begin{beamercolorbox}[sep=8pt,center]{part title}
    \usebeamerfont{subsection title}\insertsubsection\par
  \end{beamercolorbox}
}
\AtBeginPart{
  \frame{\partpage}
}
\AtBeginSection{
  \ifbibliography
  \else
    \frame{\sectionpage}
  \fi
}
\AtBeginSubsection{
  \frame{\subsectionpage}
}
\usepackage{amsmath,amssymb}
\usepackage{lmodern}
\usepackage{iftex}
\ifPDFTeX
  \usepackage[T1]{fontenc}
  \usepackage[utf8]{inputenc}
  \usepackage{textcomp} % provide euro and other symbols
\else % if luatex or xetex
  \usepackage{unicode-math}
  \defaultfontfeatures{Scale=MatchLowercase}
  \defaultfontfeatures[\rmfamily]{Ligatures=TeX,Scale=1}
\fi
% Use upquote if available, for straight quotes in verbatim environments
\IfFileExists{upquote.sty}{\usepackage{upquote}}{}
\IfFileExists{microtype.sty}{% use microtype if available
  \usepackage[]{microtype}
  \UseMicrotypeSet[protrusion]{basicmath} % disable protrusion for tt fonts
}{}
\makeatletter
\@ifundefined{KOMAClassName}{% if non-KOMA class
  \IfFileExists{parskip.sty}{%
    \usepackage{parskip}
  }{% else
    \setlength{\parindent}{0pt}
    \setlength{\parskip}{6pt plus 2pt minus 1pt}}
}{% if KOMA class
  \KOMAoptions{parskip=half}}
\makeatother
\usepackage{xcolor}
\IfFileExists{xurl.sty}{\usepackage{xurl}}{} % add URL line breaks if available
\IfFileExists{bookmark.sty}{\usepackage{bookmark}}{\usepackage{hyperref}}
\hypersetup{
  pdftitle={1. Modules},
  hidelinks,
  pdfcreator={LaTeX via pandoc}}
\urlstyle{same} % disable monospaced font for URLs
\newif\ifbibliography
\setlength{\emergencystretch}{3em} % prevent overfull lines
\providecommand{\tightlist}{%
  \setlength{\itemsep}{0pt}\setlength{\parskip}{0pt}}
\setcounter{secnumdepth}{-\maxdimen} % remove section numbering
\usepackage[all]{xy}
\input macros.tex
\ifLuaTeX
  \usepackage{selnolig}  % disable illegal ligatures
\fi

\title{1. Modules}
\author{}
\date{}

\begin{document}
\frame{\titlepage}

\hypertarget{modules}{%
\subsection{Modules}\label{modules}}

\begin{frame}{Modules}
\begin{itemize}
\tightlist
\item
  Modules are to rings as vector spaces are to fields.
\item
  Modules are to rings as sets with group actions are to groups.
\end{itemize}
\end{frame}

\begin{frame}{Definition of (left) modules}
\protect\hypertarget{definition-of-left-modules}{}
\textbf{Definition:} Let \(R\) be a ring (for now, not necessarily
commutative and not necessarily having a unit). A \emph{left
\(R\)-module} is an abelian group \(M\) together with a map
\(R\times M\to M\) (written \((r,m)\mapsto rm\)) such that:

\begin{itemize}
\tightlist
\item
  \(r(m_1+m_2)=rm_1+rm_2\)
\item
  \((r_1+r_2)m = r_1 m + r_2 m\)
\item
  \(r_1 (r_2 m) = (r_1 r_2) m\)
\end{itemize}

If \(R\) has a unit element \(1\), we also require \(1m=m\) for all
\(m\in M\).
\end{frame}

\begin{frame}{Right modules}
\protect\hypertarget{right-modules}{}
A right module is defined by a map \(M\times R\to M\) and written
\((m,r)\mapsto mr\) and satisfying the property \[
(m r_1)r_2 = m(r_1 r_2).
\]

If \(R\) is not commutative, these really are different, since for a
left module:

\begin{itemize}
\tightlist
\item
  \(r_1 r_2\) acts by ``first \(r_2\), then \(r_1\)
\end{itemize}

while for a right module

\begin{itemize}
\tightlist
\item
  \(r_1 r_2\) acts by ``first \(r_1\), then \(r_2\).''
\end{itemize}
\end{frame}

\begin{frame}{Left and Right modules}
\protect\hypertarget{left-and-right-modules}{}
If \(R\) is commutative, and \(M\) is a left \(R\)-module, then we can
define a right \(R\) module \(M'\) with the same underlying abelian
group \(M\) and by defining \(m' r=(r m)'\). This works because

\[
(m'r_1) r_2 = (r_1 m)'r_2 = (r_2(r_1 m))'=((r_2 r_1)m)' =((r_1 r_2)m)' = m'(r_1 r_2)
\]
\end{frame}

\begin{frame}[fragile]{Remarks}
\protect\hypertarget{remarks}{}
\begin{block}{Vector spaces}
\protect\hypertarget{vector-spaces}{}
If \(R\) is a field, then a left (or right) \(R\)-module is the same as
a vector space.
\end{block}

\begin{block}{Another definition}
\protect\hypertarget{another-definition}{}
If \(M\) is an abelian group, and \(R\) is a ring, then a left
\(R\)-module structure on \(M\) is the same as a ring map
\[R\to \End (M)\]. If \(\phi_r\) is the endomorphism associated to
\(r\in R\), then \(rm=\phi_{r}(m)\). The associativity comes from
defining the ring structure on \[\End (M)\] as the usual composition of
functions: \[
\phi_{r_1 r_2}=\phi_{r_1}\circ\phi_{r_2}.
\]

Testing \(\GL(\R)\)

\begin{verbatim}
<a href="slides/01-modules.html"> View as slides </a>
\end{verbatim}
\end{block}
\end{frame}

\end{document}
